\documentclass[aspectratio=169, lecture, amberg]{OTHAWbeamer}
\let\Tiny=\tiny  
\usepackage{ngerman}
\usepackage{amssymb}
\usepackage[utf8]{inputenc}
\usepackage{etex}
\usepackage{biblatex}
\usepackage{csquotes}
\usepackage{comment}
\usepackage{amsmath}
\usepackage{multirow}
\usepackage{makecell}
\usepackage{booktabs}
\usepackage{tabularx}

\usepackage{arydshln}


\addbibresource{references.bib}
\setbeameroption{show notes on second screen=right}
\setbeamerfont{note page}{size=\scriptsize}
\title[Forschungsseminar]{One-Step Image Translation with Text-to-Image Models}
\subtitle{Forschungsseminar}
\author[Schmidt]{Fabian Schmidt}
\place{OTH Amberg-Weiden}
\date{\today}

\email{f.schmidt3@oth-aw.de}

\begin{document}
\maketitle

% ---------- Begin Präsentation ----------
\frame{
\frametitle{Table of Contents}
\begin{enumerate}
    \item Introduction
    \item Related Work
    \item Terminology
    \item Method
    \item Experiments
    \item Discussion and Limitations
    \item Live Demo
\end{enumerate}
\tableofcontents
}

\begin{frame}
    \frametitle{Introduction}
    \framesubtitle{Problems with Diffusion Models}
    
    \begin{columns}
        \column{0.5\textwidth}
        \centering
        \includegraphics[width=0.9\textwidth]{images/GANs_Diffusion_Autoencoders.png}
    
        \column{0.5\textwidth}
        \centering
        \includegraphics[width=\textwidth]{images/Generation-with-Diffusion-Models-ezgif.com-webp-to-jpg-converter.jpg}
      \end{columns}  
      \tiny{\footnotemark \url{https://developer.nvidia.com/blog/improving-diffusion-models-as-an-alternative-to-gans-part-1/}}
      \tiny{\footnotemark \url{https://miro.medium.com/v2/resize:fit:720/format:webp/1*RDPhd2dvmHE4UrAP-QHb9w.png}}
    \end{frame}
    \note{
        In dieser Arbeit werden zwei limitierende Faktoren von Diffusion Models angesprochen:\\
        \begin{itemize}
            \item Ihre lange Inference Zeit durch das Interative Denoising
            \item Die Notwendigkeit von Paired Daten
        \end{itemize}
    }
    
    % ---------- Proposed solutions ----------
    \begin{frame}
    \frametitle{Introduction}
    \framesubtitle{Proposed solutions}
    \begin{itemize}
        \item One-step image-to-image translation method for paired and unpaired settings
        \item Reduce number of inference steps to 1
        \item Trainable without image pairs
        \item Adapt pre-trained text-conditional one-step diffusion model to new domains via adversarial learning
    \end{itemize}
\end{frame}
\note{
    \begin{itemize}        
        \item Um das zu erreichen wird ein One-Step Image-to-Image Translation Ansatz vorgeschlagen, der sowohl in gepaarten als auch ungepaarten Einstellungen funktioniert.\\
        \item Dieser Ansatz reduziert die Anzahl der Inferenzschritte auf 1 und kann ohne Bildpaare trainiert werden.\\
        \item Außerdem kann ein vortrainiertes Text-Conditional One-Step Diffusion Model durch adversariales Lernen an neue Domänen angepasst werden.
        \item Es werden zwei Modelle traniert und mit bestehenden Methoden verglichen. 
        \item CycleGAN-Turbo für Unpaired Image-to-Image Translation
        \item Pix2Pix-Turbo für Paired Image-to-Image Translation
    \end{itemize}
}

\begin{frame}
    \frametitle{Related Work}
    \framesubtitle{Image-to-Image translation}
    Paired Image Translation
    \begin{itemize}
        \item e.g. GLIGEN \cite{li2023gligen}, T2I-Adapter \cite{mou2023t2i}, ControlNet \cite{zhang2023adding}
        \item requires large number of training pairs
        \item slow inference
    \end{itemize}
    Unpaired Image Translation
    \begin{itemize}
        \item GAN- or diffusion-based methods \cite{cyclediffusion} \cite{su2022dual} \cite{sasaki2021unitddpm}
        \item require training from scratch on new domains   
    \end{itemize}
\end{frame}
\note{
    \begin{itemize}
        
        \item Bei Image-to-Image translation versucht das Modell ein von einer source Domain in eine traget Domain zu übersetzen. Hierzu wird eine Kombination von reconstruction und adversarial loss verwendet.
        \item Aktuelle Arbeiten wie GLIGEN, T2I-Adapter und ControlNet bauen auf vortrainierten text-to-image modellen auf.
        \item Diese Methoden benötigen jedoch eine große Anzahl an Trainingsdaten und haben eine langsame Inferenzzeit.\\
        \item Es gibt auch Arbeiten die unpaired Image Translation verwenden. Diese Modelle basieren auf GANs oder Diffusion und benötigen ein komplettes retraining wenn sie auf neue Domains angewendet werden sollen.\\
        \item Was Paired und Unpaired Daten sind, kommt gleich. 
    \end{itemize}
    
    }
    
    % ---------- Text-to-Image models ----------
\begin{frame}
    \frametitle{Related Work}
    \framesubtitle{Text-to-Image models}
    \begin{itemize}
        \item Large-scale text-conditioned models have enhanced image quality and diversity by training on vast datasets \cite{schuhmann2022laion5b, kakaobrain2022coyo-700m}
        \item Zero-shot methods for editing real images use pre-trained text-to-image models, such as SDEdit \cite{meng2022sdedit}
        \item Despite impressive results, these methods face challenges in complex scenes with multiple objects.
    
    \end{itemize}
\end{frame}
\note{
    \begin{itemize}
        \item Große Text-to-Image Modelle haben gute Bildquilität und Vielfalt durch das Training auf rießigen Datensätzen.
        \item Zero-shot Methoden für die Bearbeitung von echten Bildern verwenden vortrainierte Text-to-Image Modelle wie SDEdit.        
        \item SDEdit bearbeitet reale Bilder, indem es dem Eingabebild Rauschen hinzufügt und es anschließend mit einem vorher trainierten Modell dem Prompt entsprechend entrauscht
        \item Trotz beeindruckender Ergebnisse haben diese Methoden Herausforderungen in komplexen Szenen mit mehreren Objekten.
        \item Bei den Experimenten wird das hier entwickelte Modell unteranderem mit SDEdit verglichen.
    \end{itemize}
}
    
% ---------- One-step generative models ----------
\begin{frame}
    \frametitle{Related Work}
    \framesubtitle{One-step generative models}
    To expedite diffusion model inference, recent works focus on:
    \begin{itemize}
        \item reducing the number of sampling steps using ODE solvers \cite{karras2022elucidating, lu2022dpmsolver}
        \item distilling slow multistep teacher models into fast few-step student models \cite{meng2022sdedit, salimans2022progressive}
        \item Using GANs directly for text-to-image synthesis \cite{kang2023scaling, sauer2023stylegant}
    \end{itemize}
    
    This work presents the first one-step conditional model that use both text and conditioning images.
\end{frame}
\note{
    \begin{itemize}
        \item Statt vielen denoising Schritten im Diffusion wird ein einziger Schritt verwendet.\\
        \item Andere Arbeiten verwenden ODE Solver um die Anzahl der Schritte zu reduzieren.\\
        \item Es gibt auch Arbeiten die langsame Modelle in schnelle Modelle umwandeln.\\
        \item GANs werden auch direkt für Text-to-Image Synthese verwendet.\\
        \item Diese Arbeit stellt das erste One-Step Conditional Model vor, das sowohl Text als auch Konditionierungs-Bilder verwendet.
        
    \end{itemize}
    
}

\begin{frame}
    \frametitle{Terminology}
    \framesubtitle{Generative Adversarial Networks(GAN)}
    \begin{figure}
        \centering
        \includegraphics[width=0.65\linewidth]{images/blog_gan.png}
        \caption{GAN training process}
    \end{figure}
    \tiny{\footnotemark \url{http://www.lherranz.org/2018/08/07/imagetranslation/}}
\end{frame}
\note{
    \begin{itemize}
        \item Generative Adversarial Networks: Zwei Netzwerke, Generator und Diskriminator, die gegeneinander trainiert werden        
            \begin{itemize}
                \scriptsize\item Generator: versucht die verteilung der Trainingsdaten zu lernen ohne diese zu kennen
                \scriptsize\item Diskriminator: Unterscheidet zwischen echten und generierten Bildern        
            \end{itemize}            
        \item Loss Function: Min-Max Game zwischen Generator und Diskriminator 
    \end{itemize}
}

\begin{frame}
    \frametitle{Terminology}
    \framesubtitle{Generative Adversarial Networks(GAN) - Loss Function}
    \begin{block}{GAN Loss Function}
        
        \begin{equation}
            \min_G \max_D \mathcal{L}(D,G) = \mathbb{E}_{x \sim p_{\text{data}}(x)}[\log D(x)] + \mathbb{E}_{z \sim p_z(z)}[\log(1 - D(G(z)))]
        \end{equation}
    
    \end{block}
    \tiny{\footnotemark \url{http://www.lherranz.org/2018/08/07/imagetranslation/}}
\end{frame}
\note{       
    
    \begin{itemize}
        \item x: echtes bild
        \item z: Zufälliger Input für den Generator
        \item $p_{\text{data}}(x)$: Wahrscheinlichkeitsverteilung der echten Daten
        \item $p_z(z)$: Wahrscheinlichkeitsverteilung des zufälligen Inputs
        \item $D(x)$: Wahrscheinlichkeit, dass x ein echtes Bild ist
        \item $G(z)$: Generiertes Bild                        
    \end{itemize}
    Diskriminator wird trainiert um die Wahrscheinlichket, dass korrekte Label zu erkennen zu maximieren.\\
    
    Probleme: Früh im Training ist der Generator noch sehr schlecht, deswegen wird der $log(1 - D(G(z)))$ Term sehr groß und der Gradient verschwindet.\\
    In Praxis wird $logD(G(z))$ verwendet, um dieses Problem zu umgehen.\\ 
    
    Weiterentwicklungen: DCGANs, Wasserstein Distance als Loss Function oder PG-GANs nur Name dropping

}
    
% ---------- CycleGAN ----------
\begin{frame}
    \frametitle{Terminology}
    \framesubtitle{CycleGAN}
    \begin{figure}
        \centering
        \includegraphics[width=0.78\linewidth]{images/blog_cyclegan_h2z2h-768x333.png}
        \caption{CycleGAN Architecture}
    \end{figure}
    \tiny{\footnotemark \url{http://www.lherranz.org/2018/08/07/imagetranslation/}}
\end{frame}
    
    
        
% ---------- UNet and Skip Connections ----------
\begin{frame}
    \frametitle{Terminology}
    \framesubtitle{UNet and Skip Connections}
    \begin{figure}
        \centering
        \includegraphics[width=0.6\linewidth]{images/Group14.jpg}
        \caption{Architecture}
    \end{figure}
\end{frame}
\note{
    \begin{itemize}
        \item UNet: Convolutional Neural Network, das für Bildsegmentierung verwendet wird
        \begin{itemize}
            \item Encoder Pfad: Mehrere Blöcke von convolutional layers mit ReLU activation und max pooling. Reduziert die Dimensionalität des Inputs
            \item Decoder Pfad: Mehrere Blöcke von convolutional layers mit RelU activation und upconvolution. Erhöht die Dimensionalität des Inputs. Außerdem Concatenation mit entsprechenden Encoder Pfad
            \item Skip Connections: Verbindung zwischen Encoder und Decoder Pfad. Hilft details zu erhalten, die im Encoder verloren gehen würden
        \end{itemize}
    \end{itemize}
}
    
% ---------- LoRA Weights ----------
\begin{frame}
    \frametitle{Terminology}
    \framesubtitle{LoRA Weights}
    \begin{figure}
        \centering
        \includegraphics[width=0.4\linewidth]{images/Bildschirmfoto vom 2024-04-15 10-21-59.png}
        \caption{\textbf{Lo}w \textbf{R}ank \textbf{A}daption}
    \end{figure}
\end{frame}

% ---------- Paired vs unpaired data ----------
\begin{frame}
\frametitle{Terminology}
\framesubtitle{Paired vs unpaired data}
\begin{columns}
    \column{0.5\textwidth}
    \centering    
    \begin{figure}
        \includegraphics[width=0.75\textwidth]{images/blog_unpairedimagetranslation2.png}
        \caption{Unpaired Data}
    \end{figure}

    \column{0.5\textwidth}    
    \centering    
    \begin{figure}        
        \includegraphics[width=0.75\textwidth]{images/blog_pairedimagetranslation.png}
        \caption{Paired Data}
    \end{figure}
    \end{columns}
\end{frame}
\note{
    \begin{itemize}
        \item Paired data: Jedes Bild in Domain X hat ein korrespondierendes Bild in Domain Y
        \item Unpaired data: Es gibt keine direkte Zuordnung zwischen den Bildern in Domain X und Domain Y
    \end{itemize}

}


\begin{frame}
\frametitle{Method}
\framesubtitle{Adding Conditioning Input}
\begin{figure}
    \centering
    \includegraphics[width=1\linewidth]{images/Bildschirmfoto vom 2024-04-14 10-57-39.png}
    \caption{Conflicts between noise and conditional input}
\end{figure}
\end{frame}

% ---------- Preserving Input Details ----------
\begin{frame}
\frametitle{Method}
\framesubtitle{Preserving Input Details}
\begin{figure}
    \centering
    \includegraphics[width=0.9\linewidth]{images/Bildschirmfoto vom 2024-04-14 11-06-40.png}
    \caption{Skip Connections help retain details}
\end{figure}
\end{frame}

% ---------- Preserving Input Details ----------
\begin{frame}
\frametitle{Method}
\framesubtitle{Preserving Input Details}
\begin{figure}
    \centering
    \includegraphics[width=1\linewidth]{images/method.jpg}
    \caption{Model Architecture}
\end{figure}
\end{frame}

% ---------- Unpaired Training ----------

\begin{frame}
    \frametitle{Method}
    \framesubtitle{Unpaired Training}
    \begin{align*}
    \text{Goal:} &\text{ Convert images from } \mathcal{X} \subset \mathbb{R} ^{H \times W \times 3} \
    \text{ to } \mathcal{Y} \subset \mathbb{R} ^{H \times W \times 3} \\
    &\text{ given an unpaired dataset } \mathcal{X} = {x \in \mathcal{X} } \text{ and } \mathcal{Y} = {y \in \mathcal{Y} } \\
    &\text{ using one network } G \text{ and two translations } G(x, c_y): \mathcal{X} \rightarrow \mathcal{Y} \
    \text{ and } G(y, c_x): \mathcal{Y} \rightarrow \mathcal{X}
    \end{align*}
    \end{frame}

\begin{frame}
\frametitle{Method}
\framesubtitle{Unpaired Training}
\begin{block}{Cycle consistency with perecptual loss}
    \begin{equation}
        \mathcal{L}_{\text{cycle}}(G, F) = \mathbb{E}_x [ \mathcal{L}_\text{rec} (G(G(x,c_Y), c_X), x) ] + \mathbb{E}_y [ \mathcal{L}_\text{rec} (G(G(y,c_X), c_Y), y) ]
    \end{equation}
\end{block}
with $\mathcal{L}_{\text{rec}}$ as combination of L1 and LPIPS \cite{zhang2018unreasonable}
\begin{block}{Adversarial loss}
    \begin{align}
        \mathcal{L}_{\text{GAN}} &= \mathbb{E}_{y} [\log D_Y(y)] + \mathbb{E}_{x} [\log(1 - D_Y(G(x,c_Y)))] \\
        &+ \mathbb{E}_{x} [\log D_X(x)] + \mathbb{E}_{Y} [\log(1 - D_X(G(y,c_X)))]
    \end{align}
\end{block}
\end{frame}

\begin{frame}
\frametitle{Method}
\framesubtitle{Unpaired Training}
\begin{block}{Identity regularization loss}
    \begin{equation}
        \mathcal{L} _{\text{idt}} = \mathbb{E} _y [ \mathcal{L}_{\text{rec}}(G(y,c_Y),y)] + \mathbb{E}_x [ \mathcal{L}_{\text{rec}}(G(x,c_X),x)]
    \end{equation}
\end{block}

\begin{block}{Full objective}
    \begin{equation}
        \arg \underset{G}{\min} \mathcal{L}_{\text{cycle}} + \lambda _{\text{idt}} \mathcal{L}_{\text{idt}} + \lambda_{\text{GAN}}\mathcal{L}_{\text{GAN}}
    \end{equation}
\end{block}
    

\end{frame}


% ---------- Extensions ----------
\begin{frame}
\frametitle{Method}
\framesubtitle{Extensions - Paired Training}
\begin{itemize}
    \item Adaptation of network G to paired setting, like edge-to-image or sketch-to-image, called pix2pix-Turbo
    \item new translation function $G(x,c): X \rightarrow Y$ where $X$ is source domain, $Y$ target domain and $c$ conditioning input
\end{itemize}
\end{frame}

\begin{frame}
    \frametitle{Method}
    \framesubtitle{Extensions - Generating diverse output}
    Introduction of interpolation coefficient $\gamma$ \newline 
    Three changes to the Architecture (1/3):
    \begin {itemize}
        \item Generator function $G(x,z,\gamma)$ combines noise z and encoder output like so: $\gamma G_{\text{enc}}(x) + (1 - \gamma) z$
        \item Output as U-Net input
    \end{itemize}
\end{frame}

\begin{frame}
    \frametitle{Method}
    \framesubtitle{Extensions - Generating diverse output}
    Three changes to the Architecture (2/3):
    \begin {itemize}
        \item Scale LoRA weights and skip connections according to $\theta = \theta_0 + \gamma \Delta \theta$  
        \item where $\theta_0$ and $\Delta \theta$ denote the original weights and new weights.
    \end{itemize}
\end{frame}

\begin{frame}
    \frametitle{Method}
    \framesubtitle{Extensions - Generating diverse output}
    Three changes to the Architecture (3/3):
    \begin {itemize}
        \item Scale reconstruction loss according to $\gamma$: $\mathcal{L}_{\text{diverse}} = \mathcal{L}_{x,y,z,\gamma} \gamma\mathcal{L}_{\text{rec}}(G(x,z,\gamma),y)]$
        \item $\gamma = 0$ corresponds to default stochastic behavior of pretrained model, in this cas reconstruction loss is not enforced
        \item $\gamma = 1$ corresponds to deterministic translation from previoues seections
    \end{itemize}
\end{frame}


\begin{frame}
\frametitle{Experiments - Paired Image Translation}
\framesubtitle{Baselines}
Training details:
\begin{itemize}
    \item 330MB of trainable parameters for unpaired models(LoRA weights, zero-conv layer, first conv layer of U-Net)
    \item Adam Optimizer with learning rate: 1e-6, batch size:8, $\lambda _{\text{idt}} = 1$, $\lambda _{\text{GAN}} = 0.5$
\end{itemize}
Datasets:
\begin{table}
    \centering
    \begin{tabular}{|c|c|c|c|}
        Task & Images Source & Images Target & Dataset \\
        \hline
        Horse $\leftrightarrow$ Zebra & 939 & 1177 & ImageNet \cite{5206848}\\
        Winter $\leftrightarrow$ Summer & 854 & 1273 & Flickr \cite{zhu2020unpaired} \\
        Day $\leftrightarrow$ Night & Day subset & Night subset & BDD100k \cite{yu2020bdd100k}\\
        Clear $\leftrightarrow$ Foggy & 12454 & 572 from & BDD100k and DENSE \cite{bijelic2020seeing}
    \end{tabular}       
\end{table}
\end{frame}

\begin{frame}
    \frametitle{Experiments - Paired Image Translation}
    \framesubtitle{Baselines}
    Evaluation Protocol:
    \begin{itemize}
        \item match data distribution of target domain $\rightarrow$ FID \cite{heusel2018gans}
        \item preserve input image structure in translated output $\rightarrow$ DINO \cite{tumanyan2022splicing}
        \item Inference runtime using a single NVIDIA RTX A6000 GPU
        \item human preceptual study
    \end{itemize}
    
    
\end{frame}

% ---------- Comparison to Unpaired Methods ----------
\begin{frame}
    \frametitle{Experiments - Paired Image Translation}
    \framesubtitle{Comparison to Unpaired Methods}
    \begin{figure}
        \centering
        \includegraphics[width=0.85\linewidth]{images/horse_zebra.png}
        
    \end{figure}
\end{frame}

\begin{frame}
    \frametitle{Experiments - Paired Image Translation}
    \framesubtitle{Comparison to Unpaired Methods}
    \begin{figure}
        \centering
        \includegraphics[width=1\linewidth]{images/horse_zebra_table.png}
        
    \end{figure}
\end{frame}

\begin{frame}
    \frametitle{Experiments - Paired Image Translation}
    \framesubtitle{Comparison to Unpaired Methods}
    \scriptsize\centering\begin{tabular}{c c c c c c c c c c}
        \Xhline{2\arrayrulewidth}
        \multirow{2}{*}{\textbf{Method}} & \multirow{2}{*}{\textbf{\makecell{Inference \\ time}}}  & \multicolumn{2}{c}{\textbf{Horse $\rightarrow$ Zebra}} & \multicolumn{2}{c}{\textbf{Zebra $\rightarrow$ Horse}} & \multicolumn{2}{c}{\textbf{Summer $\rightarrow$ Winter}} & \multicolumn{2}{c}{\textbf{Winter $\rightarrow$ Summer}}\\
        \cline{3-4} \cline{5-6} \cline{7-8} \cline{9-10}
        & & FID $\downarrow$ & \makecell{DINO \\ Struct. $\downarrow$} & FID $\downarrow$ & \makecell{DINO \\ Struct. $\downarrow$} & FID $\downarrow$ & \makecell{DINO \\ Struct. $\downarrow$} & FID $\downarrow$ & \makecell{DINO \\ Struct. $\downarrow$} \\
        \cline{1-10}
        CycleGAN \cite{zhu2020unpaired} & 0.01s & 74.9 & 3.2 & 133.8 & 2.6 & 62.9 & 2.6 & 66.1 & 2.3\\ 
        CUT \cite{park2019semantic} & 0.01s & 43.9 & 6.6 & 196.7 & 2.5& 72.1& 2.1 & 68.5 & 2.1\\
        \cline{1-10}
        SDEdit \cite{meng2022sdedit} & 1.56s & 77.2 & 4.0 & 198.5 & 4.6 & 66.1 & 2.1 & 76.9 & 2.1 \\
        Plug-and-Play \cite{tumanyan2022plugandplay} & 7.57s & 57.3 & 5.2 & 152.4& 3.8 & 67.3 & 2.8 & 73.3 & 2.6\\
        Pix2Pix-Zero & 14.75s & 81.5 & 8.0 & 147.4 & 7.8 & 68.0 & 3.0 & 93.4 & 4.3 \\
        Cycle-Diffusion & 3.72s & \textbf{38.6} & 6.0 & 132.5 & 5.8 & 64.1 & 3.6 & 70.3 & 3.6 \\
        DDIB & 4.37s & 44.4 & 13.1 & 163.3 & 11.1 & 90.8 & 7.2 & 88.9 & 6.8\\
        InstructPix2Pix & 3.86s & 51.0 & 6.8 & 141.5 & 7.0 & 68.3 & 3.7 & 85.6 & 4.4\\
        \cline{1-10}
        \textbf{CycleGAN-Turbo} & 0.13s & 41.0 & \textbf{2.1} & \textbf{127.5} & \textbf{1.8} & \textbf{56.3} & \textbf{0.6} & \textbf{60.7} & \textbf{0.6}\\
        \Xhline{2\arrayrulewidth}
        \end{tabular}
\end{frame}

\begin{frame}
\frametitle{Experiments - Paired Image Translation}
\framesubtitle{Comparison to Unpaired Methods}
\begin{figure}
    \centering
    \includegraphics[width=0.5\linewidth]{images/day_night.png}
    
\end{figure}
\end{frame}

\begin{frame}
    \frametitle{Experiments - Paired Image Translation}
    \framesubtitle{Comparison to Unpaired Methods}
    \begin{figure}
        \centering
        \includegraphics[width=1\linewidth]{images/day_night_table.png}
        
    \end{figure}
\end{frame}

\begin{frame}
    \frametitle{Experiments - Paired Image Translation}
    \framesubtitle{Comparison to Unpaired Methods}
    \begin{figure}
        \centering
        \includegraphics[width=1\linewidth]{images/human-pref.png}
        
    \end{figure}
\end{frame}


% ---------- Ablation Study ----------
\begin{frame}
\frametitle{Experiments - Paired Image Translation}
\framesubtitle{Ablation Study}
\begin{figure}
    \centering
    \includegraphics[width=1\linewidth]{images/ablation_horse_zebra.png}
    
\end{figure}
\end{frame}

\begin{frame}
    \frametitle{Experiments - Paired Image Translation}
    \framesubtitle{Ablation Study}
    \begin{figure}
        \centering
        \includegraphics[width=0.55\linewidth]{images/ablation_images.png}
        
    \end{figure}
    \end{frame}
% ---------- Extensions ----------
\begin{frame}
\frametitle{Experiments - Unpaired Image Translation}
\framesubtitle{Training Details}
\begin{block}{Loss function}
    \begin{align}
        \arg \underset{G}{\min} \mathcal{L}_{\text{rec}} + \lambda _{\text{clip}} \mathcal{L}_{\text{CLIP}} + \lambda_{\text{GAN}}\mathcal{L}_{\text{GAN}}
    \end{align}
\end{block}
with $\mathcal{L}_{\text{rec}}$ = L2-Norm + LPIPS, $\lambda_{\text{clip}} = 4$ and $\lambda_{\text{GAN}} = 0.4$
\end{frame}

\begin{frame}
    \frametitle{Experiments - Unpaired Image Translation}
    \framesubtitle{Training Details}
        Edge-to-Image:
        \begin{itemize}
            \item Canny Edge Detector with random threshold
            \item Adam Optimizer with learning rate: 1e-5, batch size: 40, Steps: 7500
        \end{itemize}
        Sketch-to-Image:
        \begin{itemize}
            \item Synthetic sketches with multiple augmentations
            \item Initialized with Edge-to-Image model and fine-tuned for 5000 steps with same Optimizer
        \end{itemize}
\end{frame}

\begin{frame}
    \frametitle{Experiments - Unpaired Image Translation}
    \framesubtitle{Comparison to Unpaired Methods}
    \begin{figure}
        \centering
        \includegraphics[width=0.5\linewidth]{images/unpaired_comp1.png}
        
    \end{figure}
\end{frame}

\begin{frame}
    \frametitle{Experiments - Unpaired Image Translation}
    \framesubtitle{Comparison to Unpaired Methods}
    \begin{figure}
        \centering
        \includegraphics[width=1\linewidth]{images/unpaired_comp2.png}
            
    \end{figure}
\end{frame}


\begin{frame}
    \frametitle{Discussion and Limitations}
    \framesubtitle{Discussion}
    \begin{itemize}
        \item one-step pre-trained models can serve as a backbone model for many image synthesis tasks
        \item Adapting the models can be achieved through GAN objectives without multi-step diffusion training
        \item model training requires a small number of additional trainable parameters
    \end{itemize}
\end{frame}
    
% ---------- Limitations ----------
\begin{frame}
    \frametitle{Discussion and Limitations}
    \framesubtitle{Limitations}
    \begin{itemize}
        \item cannot specify strength of guidance as SD-Turbo does not use classifier-free guidance
        \item does not support negative prompt
        \item training is memory intensive
    \end{itemize}
    
\end{frame}

% ---------- End ----------
\begin{frame}
\frametitle{The End}
\begin{center}
\scalebox{2}{Questions?}
\end{center}
\end{frame}

\begin{frame}[allowframebreaks]
\frametitle{References}
\printbibliography
\end{frame}
\end{document}