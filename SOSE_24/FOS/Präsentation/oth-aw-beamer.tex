\documentclass[aspectratio=169, lecture, amberg]{OTHAWbeamer}
\let\Tiny=\tiny  % remove annoying warning 2
% other formats possible, e.g. aspectratio=43 or aspectration=169
% lecture argument adds the subtitle to the footline
% instead of amberg, the option weiden oder amwen can be given. This changes the title backround

\usepackage{ngerman}
\usepackage{amssymb}

\usepackage[utf8]{inputenc}
%\usepackage{tikz}
\usepackage{etex}

\title[Forschungsseminar]{One-Step Image Translation with Text-to-Image Models}
\subtitle{Forschungsseminar}
\author[Schmidt]{Fabian Schmidt}
\place{OTH Amberg-Weiden}
\date{\today}
%\institute{Fakultät XXX}

\email{f.schmidt3@oth-aw.de}

\begin{document}
\maketitle

\frame{
\frametitle{Tabel of Contents}
\begin{enumerate}
    \item Introduction
    \item Related Work
    \item Terminology
    \item Method
    \item Experiments
    \item Discussion and Limitations
    \item Live Demo
\end{enumerate}
\tableofcontents
}
\begin{frame}
\frametitle{Introduction}
\framesubtitle{Problems with Diffusion Models}

\begin{columns}
    \column{0.5\textwidth}
    \centering
    \includegraphics[width=\textwidth]{images/GANs_Diffusion_Autoencoders.png}

    \column{0.5\textwidth}
    \centering
    \includegraphics[width=\textwidth]{images/Generation-with-Diffusion-Models-ezgif.com-webp-to-jpg-converter.jpg}
  \end{columns}
\end{frame}


\begin{frame}
\frametitle{Introduction}
\framesubtitle{Proposed solutions}
\begin{itemize}
    \item One-step image-to-image translation method for paired and unpaired settings
    \item Reduce number of inference steps to 1
    \item Trainable without image pairs
    \item Adapt pre-trained text-conditional one-step diffusion model to new domains via adversarial learning
\end{itemize}
\end{frame}

\begin{frame}
\frametitle{Related Work}
\framesubtitle{Text-to-Image translation}

\end{frame}

\begin{frame}
\frametitle{Related Work}
\framesubtitle{Text-to-Image models}

\end{frame}

\begin{frame}
\frametitle{Related Work}
\framesubtitle{One-step generative models}

\end{frame}




\begin{frame}
\frametitle{Terminology}
\framesubtitle{Generative Adversarial Networks(GAN)}
\begin{figure}
    \centering
    \includegraphics[width=0.75\linewidth]{images/blog_gan.png}
    \caption{GAN training process}
\end{figure}
\end{frame}

\begin{frame}
\frametitle{Terminology}
\framesubtitle{Generative Adversarial Networks(GAN)}
\begin{block}{Generator loss}
\begin{equation}
\min_G \mathcal{L}_G = \mathbb{E}_{z \sim p_z(z)} [\log(1 - D(G(z)))]
\end{equation}



\end{block}
\begin{block}{Discriminator loss}
\begin{equation}
\max_D \mathcal{L}_D = \mathbb{E}_{x \sim p_{\text{data}}(x)} [\log D(x)] + \mathbb{E}_{z \sim p_z(z)} [\log(1 - D(G(z)))]
\end{equation}\end{block}
\end{frame}

\begin{frame}
\frametitle{Terminology}
\framesubtitle{CycleGAN}
\begin{figure}
    \centering
    \includegraphics[width=0.8\linewidth]{images/blog_cyclegan_h2z2h-768x333.png}
    \caption{CycleGAN Architecture}
\end{figure}
\end{frame}

\begin{frame}
\frametitle{Terminology}
\framesubtitle{CycleGAN}
\begin{block}{Adversarial loss for G}
\begin{equation}
\mathcal{L}_{\text{GAN}}(G, D_Y, X, Y) = \mathbb{E}_{y \sim p_{\text{data}}(y)} [\log D_Y(y)] + \mathbb{E}_{x \sim p_{\text{data}}(x)} [\log(1 - D_Y(G(x)))]
\end{equation}
\end{block}

\begin{block}{Adversarial loss for F}
\begin{equation}
\mathcal{L}_{\text{GAN}}(F, D_X, Y, X) = \mathbb{E}_{x \sim p_{\text{data}}(x)} [\log D_X(x)] + \mathbb{E}_{y \sim p_{\text{data}}(y)} [\log(1 - D_X(F(y)))]
\end{equation}
\end{block}

\begin{block}{Cycle consistency loss}
\begin{equation}
\mathcal{L}_{\text{cycle}}(G, F) = \mathbb{E}_{x \sim p_{\text{data}}(x)} [\lVert F(G(x)) - x \rVert_1] + \mathbb{E}_{y \sim p_{\text{data}}(y)} [\lVert G(F(y)) - y \rVert_1]
\end{equation}
\end{block}
\end{frame}

\begin{frame}
\frametitle{Terminology}
\framesubtitle{CycleGAN}
\begin{block}{Identity loss}
\begin{align}
\mathcal{L}_{\text{identity}}(G, F, X, Y) & = \mathbb{E}_{y \sim p_{\text{data}}(y)} [\lVert G(y) - y \rVert_1] \\
\mathcal{L}_{\text{identity}}(F, G, Y, X) & = \mathbb{E}_{x \sim p_{\text{data}}(x)} [\lVert F(x) - x \rVert_1]
\end{align}
\end{block}

\begin{block}{Cycle loss}
\begin{equation}
\mathcal{L}(G, F, D_X, D_Y) = \mathcal{L}_{\text{GAN}}(G, D_Y, X, Y) + \mathcal{L}_{\text{GAN}}(F, D_X, Y, X) + \lambda \mathcal{L}_{\text{cycle}}(G, F) + \lambda_{\text{id}} \left( \mathcal{L}_{\text{identity}}(G, F, X, Y) + \mathcal{L}_{\text{identity}}(F, G, Y, X) \right)
\end{equation}

\end{block}
\end{frame}

\begin{frame}
\frametitle{Terminology}
\framesubtitle{Paired vs unpaired data}
\begin{columns}
    \column{0.5\textwidth}
    \centering    
    \begin{figure}
        \includegraphics[width=0.75\textwidth]{images/blog_unpairedimagetranslation2.png}
        \caption{Unpaired Data}
    \end{figure}

    \column{0.5\textwidth}    
    \centering    
    \begin{figure}        
        \includegraphics[width=0.75\textwidth]{images/blog_pairedimagetranslation.png}
        \caption{Paired Data}
    \end{figure}
  \end{columns}
\end{frame}

\begin{frame}
\frametitle{Terminology}
\framesubtitle{UNet and Skip Connections}
\begin{figure}
    \centering
    \includegraphics[width=0.75\linewidth]{images/Group14.jpg}
    \caption{Architecture}
\end{figure}
\end{frame}

\begin{frame}
\frametitle{Terminology}
\framesubtitle{LoRA Weights}
\begin{figure}
    \centering
    \includegraphics[width=0.4\linewidth]{images/Bildschirmfoto vom 2024-04-15 10-21-59.png}
    \caption{\textbf{Lo}w \textbf{R}ank \textbf{A}daption}
\end{figure}
\end{frame}

\begin{frame}
\frametitle{Terminology}
\framesubtitle{Stable Diffusion}

\end{frame}

\begin{frame}
\frametitle{Method}
\framesubtitle{Goal}
?
\end{frame}

\begin{frame}
\frametitle{Method}
\framesubtitle{Adding Conditioning Input}
\begin{figure}
    \centering
    \includegraphics[width=1\linewidth]{images/Bildschirmfoto vom 2024-04-14 10-57-39.png}
    \caption{Enter Caption}
\end{figure}
\end{frame}

\begin{frame}
\frametitle{Method}
\framesubtitle{Preserving Input Details}
\begin{figure}
    \centering
    \includegraphics[width=1\linewidth]{images/Bildschirmfoto vom 2024-04-14 11-06-40.png}
    \caption{Skip Connections help retain details}
\end{figure}
\end{frame}

\begin{frame}
\frametitle{Method}
\framesubtitle{Preserving Input Details}

\begin{figure}
    \centering
    \includegraphics[width=1\linewidth]{images/method.jpg}
    \caption{Model Architecture}
\end{figure}
\end{frame}

\begin{frame}
\frametitle{Method}
\framesubtitle{Unpaired Training}

\end{frame}

\begin{frame}
\frametitle{Method}
\framesubtitle{Extensions}

\end{frame}

\begin{frame}
\frametitle{Experiments}


\end{frame}

\begin{frame}
\frametitle{Experiments}
\framesubtitle{Comparison to Unpaired Methods}

\end{frame}

\begin{frame}
\frametitle{Experiments}
\framesubtitle{Ablation Study}

\end{frame}

\begin{frame}
\frametitle{Experiments}
\framesubtitle{Extensions}

\end{frame}

\begin{frame}
\frametitle{Discussion and Limitations}
\framesubtitle{Discussion}
\begin{itemize}
    \item one-step pre-trained models can serve as a backbone model for many image synthesis tasks
    \item Adapting the models can be achieved through GAN objectives without multi-step diffusion training
    \item model training requires a small number of additional trainable parameters
\end{itemize}
\end{frame}

\begin{frame}
\frametitle{Discussion and Limitations}
\framesubtitle{Limitations}
\begin{itemize}
    \item cannot specify strength of guidance as SD-Turbo does not use classifier-free guidance
    \item does not support negative prompt
    \item training is memory intensive
\end{itemize}

\end{frame}

\begin{frame}
\frametitle{The End}
\begin{center}
\scalebox{2}{Questions?}
\end{center}
\end{frame}
\end{document}

